\documentclass[
    ngerman,%globale Übergabe der Hauptsprache
%	logofile=example-image, %Falls die Logo Dateien nicht vorliegen
    authorontitle=true,
]{bfhbeamer}


%\usepackage[main=ngerman]{babel}

% Der folgende Block ist nur bei pdfTeX auf Versionen vor April 2018 notwendig
%\usepackage{iftex}
%\ifPDFTeX
%\usepackage[utf8]{inputenc}%kompatibilität mit TeX Versionen vor April 2018
%\fi


%Makros für Formatierungen der Doku
%Im Allgemeinen nicht notwendig!
%\let\code\texttt

\title{URL-Archiver - Intermediate Presentation}
\subtitle{Version 1.0}
\author[N. Dora \and A. Vejseli \and K. Wampfler]{N. Dora \and A. Vejseli \and K. Wampfler}
\institute{School of Engineering and Computer Science}
\titlegraphic*{\includegraphics{pictures/archive-title-image}}%is only used with BFH-graphic and BFH-fullgraphic

%Activate the output of a frame number:
\setbeamertemplate{page number in head/foot}[framenumber]

\begin{document}

%    Was ist die Problemstellung Ihres Projektes
%   (Zielsetzungen, Anforderungen, Rahmenbedingungen)
%    Kilian
%    Wie wollen Sie die Problemstellung lösen?
%    (Architektur, Datenmodell, Prozessmodell, Technologien, usw.)
%    Nicolin
%    Wie setzen Sie das Projekt mit Scrum um?
%    Abidin Vejseli

    \setbeamertemplate{title page}[BFH-fullgraphic]
    \maketitle



    \begin{frame}{Table of Content}
        \framesubtitle{Topics Covered}
        \tableofcontents
    \end{frame}

    % Chapter one "Problem Statement"


    \section{Problem Statement}
    \setbeamertemplate{section page}[BFH-ruled]
    \frame{\sectionpage}

    \begin{frame}{Problem Statement}
        \framesubtitle{The ever changing internet}
        \begin{columns} % Start the columns environment
            \begin{column}{0.6\textwidth} % Define the width of the left column
                \begin{itemize}
                    \item ever-changing internet (websites are taken down or changed)
                    \item not a problem for everyday internet users but\ldots
                    \item \ldots what about documents (e. g. theses, documentations etc.)
                    \begin{itemize}
                        \item invalid links to content-relevant information
                        \item invalid quote links
                        \item old documents may contain numerous broken hyperlinks
                    \end{itemize}
                \end{itemize}
            \end{column}
            \begin{column}{0.3\textwidth} % Define the width of the left column
                \includegraphics[width=0.6\textwidth]{pictures/broken_link}
            \end{column}
        \end{columns}
    \end{frame}

    \begin{frame}{Problem Statement}
        \framesubtitle{Simple Solution}
        \begin{columns} % Start the columns environment
            \begin{column}{0.5\textwidth} % Define the width of the left column
                \begin{itemize}
                    \item Archive the actual state of the linked websites
                    \begin{itemize}
                        \item Wayback machine
                        \item archive.today
                        \item Memento Time Travel
                        \item and many more…
                    \end{itemize}
                    \item But who has the time and motivation?
                    \begin{itemize}
                        \item \ldots to search each hyperlink
                        \item \ldots to manually archive each website
                    \end{itemize}
                \end{itemize}
            \end{column}
            \begin{column}{0.4\textwidth} % Define the width of the left column
                \includegraphics[width=0.7\textwidth]{pictures/frustrated}
            \end{column}
        \end{columns}
    \end{frame}

    \begin{frame}{Objectives}
        \framesubtitle{}
        \begin{columns} % Start the columns environment
            \begin{column}{0.5\textwidth} % Define the width of the left column
                \begin{itemize}
                    \item Develop a software that\ldots
                    \begin{itemize}
                        \item \ldots searches hyperlinks within documents
                        \item \ldots is capable of archiving linked websites
                        \item \ldots can store current and archived links in a file
                        \item \ldots is easy and intuitive to use
                    \end{itemize}
                \end{itemize}
            \end{column}
            \begin{column}{0.3\textwidth} % Define the width of the left column
                \includegraphics[width=0.6\textwidth]{pictures/lupe}
            \end{column}
        \end{columns}
    \end{frame}

    \begin{frame}{Requirements}
        \framesubtitle{}
        \begin{itemize}
            \item The software must be\ldots
            \begin{itemize}
                \item \ldots a Java application that is compatible with multiple platforms
                \item \ldots FLOSS-licensed
                \item \ldots capable of archiving websites on archive.ph or/and WayBackMachine
                \item \ldots capable of generating a CSV-file with key-value (URL, archived URL)
            \end{itemize}
        \end{itemize}

    \end{frame}

    \begin{frame}{General conditions}
        \framesubtitle{}
        \begin{columns} % Start the columns environment
            \begin{column}{0.5\textwidth} % Define the width of the left column
                \begin{itemize}
                    \item The program code should be minimal, modular, and self-explaining
                    \item The program code should be published in a git repository
                    \item The project must be carried out according to scrum
                    \item Public documents should be written in English
                    \item LaTeX should be used for the project report
                \end{itemize}
            \end{column}
            \begin{column}{0.4\textwidth} % Define the width of the left column
                \includegraphics[width=0.8\textwidth]{pictures/clean-code-1}
            \end{column}
        \end{columns}
    \end{frame}
    %-------------------------------%

    % Chapter two "Solving the Problem"


    \section{Solving The Problem}
    \setbeamertemplate{section page}[BFH-ruled]
    \frame{\sectionpage}

    \begin{frame}{Process model}
        \framesubtitle{Workflow for URL Extraction and Archiving}
        \begin{itemize}
            \item User can skip URLs, launch them, access help or quit the program at any time.
        \end{itemize}
        \vspace{0.8cm}
        \includegraphics[width=1\textwidth]{pictures/process_model-simple}
        \note[item]{Thank you for coming from my side too.}
        \note[item]{First lets talk about how an user can archive urls. }
        \note[item]{----------------------------------------------------------}
        \note[item]{For this you can see here a rough overview of how a user navigates through the application to archive URLs. }
        \note[item]{The user can open the current URL in their default browser, access the help section, or view the archived URLs at any time.}
        \note[item]{The user does not need to archive a url, he can skip the archiving process.}
        \note[item]{Now to the more technical part.}
    \end{frame}

    \begin{frame}{Architecture}
        \framesubtitle{Employing the MVC Pattern}

        \begin{columns} % Start the columns environment
            \begin{column}{0.5\textwidth} % Define the width of the left column
                \begin{itemize}
                    \item Modular design for easy extension.
                    \item Separate data, view, and control flow.
                    \item Facilitates the potential addition of a GUI.
                \end{itemize}
            \end{column}

            \begin{column}{0.5\textwidth} % Define the width of the right column
                \includegraphics[width=1\textwidth]{pictures/mvc_diagram-Detailed}
            \end{column}
        \end{columns} % End the columns environment
        \note[item]{To demonstrate how we meet the requirements for our application, which Kilian presented earlier, I will now provide a high-level overview of our structure and its individual components.}
        \note[item]{It is important to note that this is our current structure, subject to expansion and adaptation as the project advances. }
        \note[item]{To ensure our application's scalability, we utilize the MVC pattern, among other techniques. }
        \note[item]{This involves separating the business logic from the presentation using a controller. }
        \note[item]{This enables us to easly switch out the Command Line Interface with a Graphical User Interface.}
    \end{frame}

    % \begin{frame}{High Level Class Diagram}
    %     \framesubtitle{Overview}
    %     \includegraphics[width=1\textwidth]{pictures/URL_Archiver_Class_Diagram-Presentation}
    %     \note[item]{First of all: Don't worry, I will go into detail about the individual components in the following slides. }
    %
    % \end{frame}

    \begin{frame}{High Level Class Diagram}
        \framesubtitle{Main}
        \includegraphics[width=1\textwidth]{pictures/URL_Archiver_Class_Diagram-Presentation_1}
        \note[item]{I know you're thinking: didn't he say high level?}
        \note[item]{I know, I know, but I'll walk you through the diagram with the orange markers. }
        \note[item]{So just focus on the orange and stay with me.  }
        \note[item]{----------------------------------------------------------}
        \note[item]{As usual, the Main class starts the program. }
        \note[item]{It directly initializes the controller class. }
        \note[item]{The Controller class is the control instance of the entire application. It connects the business logic with the command line interface. Keyword MVC pattern.}
    \end{frame}

    \begin{frame}{High Level Class Diagram}
        \framesubtitle{CLIController}
        \includegraphics[width=1\textwidth]{pictures/URL_Archiver_Class_Diagram-Presentation_2}
        \note{Now the controller...}
        \note[item]{... checks the path and file type to avoid errors later.}
        \note[item]{... decides whether to use a FolderModel or a FileModel based on what's given.}
        \note[item]{... needs a ConsoleView to show the process on the screen for the user.}

    \end{frame}

    % \begin{frame}{High Level Class Diagram}
    %     \framesubtitle{FileModel}
    %     \includegraphics[width=1\textwidth]{pictures/URL_Archiver_Class_Diagram-Presentation_3}
    %     \note[item]{Here you can see the data model where all the business logic takes place. However, I will be going into more detail about this later on, so I will refrain from going into any more detail at this point. }
    % \end{frame}

    \begin{frame}{High Level Class Diagram}
        \framesubtitle{I18n and UserChoice}
        \includegraphics[width=1\textwidth]{pictures/URL_Archiver_Class_Diagram-Presentation_4}
        \note[item]{Our application is ready for multiple languages, thanks to the I18n class. Right now, it's set up for English, but we can easily add more languages.}
        \note[item]{We use the 'UserChoice' enum in conjunction with I18n to process user commands. While it's all in English at the moment, it's built to adapt to other languages without any hassle. }
        \note[item]{By using I18n and UserChoice, we ensure a better experience for users and make our application flexible for international use in the future.}
    \end{frame}

    \begin{frame}{High Level Class Diagram}
        \framesubtitle{Exceptions}
        \includegraphics[width=1\textwidth]{pictures/URL_Archiver_Class_Diagram-Presentation_5}
        \note[item]{Additionally, we have created our own exceptions for validating files and paths to enhance user experience.  }
    \end{frame}

    \begin{frame}{Architecture}
        \framesubtitle{Factory Pattern}
        \begin{columns} % Start the columns environment
            \begin{column}{0.6\textwidth} % Define the width of the left column
                \begin{itemize}
                    \item Scalable and robust design.
                    \item Ensures maintainability and clear code structure.
                    \item Commitment to best programming practices.
                \end{itemize}
            \end{column}
            \begin{column}{0.4\textwidth} % Define the width of the right column
                \includegraphics[width=0.7\textwidth]{pictures/URL_Archiver_Class_Diagram-FileReaderFactory}
            \end{column}
        \end{columns} % End the columns environment
        \note[item]{To enhance our application's flexibility further, we've implemented a factory class that reads and processes different file formats. }
        \note[item]{With a single call to the getFileReader() method from the FileReaderFactory, specifying the file type, we retrieve an appropriate FileReader. }
        \note[item]{Currently, our application can handle all UNICODE text files and PDFs.}
    \end{frame}

    %  \begin{frame}{Architecture}
    %      \framesubtitle{Multilanguage Support with ResourceBundle}
    %      \begin{columns} % Start the columns environment
    %          \begin{column}{0.6\textwidth} % Define the width of the left column
    %              \begin{itemize}
    %                  \item Ready for global adaptability.
    %                  \item Future-proofing architectural choice.
    %                  \item Potential to accommodate multiple languages.
    %              \end{itemize}
    %          \end{column}
    %          \begin{column}{0.4\textwidth} % Define the width of the right column
    %              \includegraphics[width=0.7\textwidth]{pictures/URL_Archiver_Class_Diagram-I18n}
    %          \end{column}
    %      \end{columns} % End the columns environment
    %  \end{frame}


% Slide for Key Components
    \begin{frame}{Key Components of the Data Model}
        \framesubtitle{Overview}
        \begin{columns} % Start the columns environment
            \begin{column}{0.5\textwidth} % Define the width of the left column
                \begin{itemize}
                    \item \textbf{FileModel}: Represents and handles individual files.
                    \item \textbf{FolderModel}: Manages a collection of FileModel instances.
                    \item \textbf{URLPair}: Links extracted URLs with their archived counterparts.
                    \item \textbf{FileReaderFactory}: Organizes and maintains URLPairs.
                    \item \textbf{URLExtractor}: Used for URL extraction.
                    \item \textbf{URLArchiver}: Used for URL archiving.
                \end{itemize}
            \end{column}
            \begin{column}{0.5\textwidth} % Define the width of the right column
                \includegraphics[width=1\textwidth]{pictures/URL_Archiver_Class_Diagram-DataModel}
            \end{column}
        \end{columns} % End the columns environment
        \note{Now, let us discuss our data model. As previously mentioned, it is comprised of the following classes:}
        \note[item]{The FileModel class manages the details of individual files, each potentially containing multiple URLPair objects, which link original URLs to their archived versions.}
        \note[item]{FolderModel is applied when a user defines a folder path, storing the folder's structure and containing a list of FileModel objects, crucial for organizing URLs into bibliographic (.BIB) files.}
        \note[item]{FileReaderFactory is our system for providing the right reader for a file based on its type, ensuring that each file is read correctly and efficiently.}
        \note[item]{With the URLExtractor, our model extracts URLs from text, and the URLArchiver takes care of the URL archiving process as instructed by the user.}
        \note[item]{This setup not only facilitates a clear division of tasks among the different components but also streamlines the process of archiving and retrieving URLs.}
    \end{frame}

    %\begin{frame}{Data Model}
    %    \framesubtitle{FolderModel Class}
    %    \begin{columns} % Start the columns environment
    %        \begin{column}{0.5\textwidth} % Define the width of the left column
    %            \begin{itemize}
    %                \item Utilized upon user-defined folder path.
    %                \item Stores folder structure within FolderModel.
    %                \item Contains list of FileModel objects.
    %                \item Essential for archiving URLs into .BIB files.
    %                \item Facilitates clear division of responsibilities.
    %            \end{itemize}
    %        \end{column}
    %        \begin{column}{0.5\textwidth} % Define the width of the right column
    %            \includegraphics[width=1\textwidth]{pictures/URL_Archiver_Class_Diagram-DataModel_1}
    %        \end{column}
    %    \end{columns} % End the columns environment
    %    \note{The FolderModel class is only used when the user specifies a path to a folder. In such a case, the structure of the folder is stored in the FolderModel, i.e. there is a list of FileModel objects. }
    %    \note[item]{This will be important later when the archive URLs are written back to a file (.BIB).}
    %    \note[item]{It also allows us to separate the responsibilities nicely.  }
    %\end{frame}
%
    %\begin{frame}{Data Model}
    %    \framesubtitle{FileModel Class}
    %    \begin{columns} % Start the columns environment
    %        \begin{column}{0.5\textwidth} % Define the width of the left column
    %            \begin{itemize}
    %                \item FileModel operates independently or within FolderModel.
    %                \item Stores list of URLPair objects.
    %                \item Utilizes FileReaderFactory to select appropriate reader.
    %                \item Extracts and saves URLs using URLExtractor.
    %                \item Archives URLs upon user's command with URLArchiver.
    %            \end{itemize}
    %        \end{column}
    %        \begin{column}{0.5\textwidth} % Define the width of the right column
    %            \includegraphics[width=1\textwidth]{pictures/URL_Archiver_Class_Diagram-DataModel_2}
    %        \end{column}
    %    \end{columns} % End the columns environment
    %    \note[item]{The FileModel class can exist without a FolderModel class.}
    %    \note[item]{This is the case when the user specifies only one file.}
    %    \note[item]{In any case, the FileModel class is currently responsible for storing a list of URLPairs objects. }
    %    \note[item]{It also uses a reader to read the text from the file; the necessary reader is provided by the FileReaderFactory, depending on the file type.}
    %    \note[item]{It then uses the URLExtractor to read the URLs from the text and store them in a URLPair. If the user wants to archive the URL, it uses the URLArchiver. }
    %\end{frame}

    \begin{frame}{Technologies I}
        \begin{columns}
            \begin{column}{0.7\textwidth}
                \textbf{Core Technologies}
                \begin{itemize}
                    \item \textbf{Java}: The primary programming language for our application.
                    \item \textbf{LaTeX}: Used for documentation and presentation.
                \end{itemize}

                \vspace{1em} % Space between core and supporting technologies

                \textbf{Supporting Technologies}
                \begin{itemize}
                    \item \textbf{JUnit 5}: Utilized for unit testing.
                    \item \textbf{PDFTextStripper (PDFBox library)}: Used for extracting text from PDF documents.
                    \item \textbf{Selenium}: Web automation tool used for tasks like inputting URLs, handling captchas, and retrieving archived URLs due to the absence of an official API from archive.today.
                    \item \textbf{Maven}: Essential for compilation, dependency management, and building the project.
                \end{itemize}
                \note[item]{To implement the project, we use multiple technologies.}
                \note[item]{The application is developed in Java, while the report, presentations, and user manual are authored in LaTeX.}
                \note[item]{For testing, we utilize JUnit 5, and for text extraction from PDFs, we rely on PDFTextStripper. }
                \note[item]{The archiving process on Archive.today is executed using Selenium to navigate the lack of an API and to address captcha challenges.}
                \note[item]{Additionally, we use Maven for building the application and managing dependencies.}
            \end{column}

            \begin{column}{0.3\textwidth}
                \includegraphics[height=0.5cm]{pictures/Java-Logo}
                \vspace{1em}
                \includegraphics[height=0.5cm]{pictures/LaTeX_logo}
                \vspace{1em}
                \includegraphics[height=0.5cm]{pictures/JUnit_5_Logo}
                \vspace{1em}
                \includegraphics[height=0.5cm]{pictures/Apache_PDFBox_logo}
                \vspace{1em}
                \includegraphics[height=0.5cm]{pictures/Selenium_logo}
                \vspace{1em}
                \includegraphics[height=0.5cm]{pictures/Apache_Maven_logo}
            \end{column}
        \end{columns}
    \end{frame}

    \begin{frame}{Technologies II}
        \begin{columns}
            \begin{column}{0.7\textwidth}
                \textbf{Archiving Services}
                \begin{itemize}
                    \item \textbf{Wayback Machine}: One of the services used by the URL-Archiver to archive URLs.
                    \item \textbf{Archive.today}: One of the services used by the URL-Archiver to archive URLs.
                \end{itemize}
                \textbf{Development Environment}
                \begin{itemize}
                    \item \textbf{JetBrains IntelliJ IDEA}: Used to develop the program and create the report and presentation in Latex.
                \end{itemize}
                \begin{itemize}
                    \item \textbf{GitHub}: A platform for code hosting and collaboration, allowing us to manage our program's development and version control.
                \end{itemize}
                \textbf{Project Management}
                \begin{itemize}
                    \item \textbf{Atlassian Jira}: A tool for tracking progress and managing our project goals and tasks.
                \end{itemize}

            \end{column}
            \begin{column}{0.3\textwidth}
                \includegraphics[height=1cm]{pictures/Wayback_Machine_logo}
                \vspace{1em}
                \includegraphics[height=1cm]{pictures/archive_today_logo}
                \vspace{1em}
                \includegraphics[height=1cm]{pictures/IntelliJ_IDEA_Icon}
                \vspace{1em}
                \includegraphics[height=1cm]{pictures/github-mark}
                \vspace{1em}
                \includegraphics[height=1cm]{pictures/jira_logo}
            \end{column}
        \end{columns}
        \note[item]{Furthermore, we rely on Archive.today and additionally on the Wayback Machine to archive URLs.}
        \note[item]{As our development environment, we use JetBrains' IntelliJ IDEA.}
    \end{frame}

    %-------------------------------%

    % Chapter two "Project Management with SCRUM"


    \section{Project Management with SCRUM}
    \setbeamertemplate{section page}[BFH-ruled]
    \frame{\sectionpage}

    \begin{frame}{Our Scrum Team}
        \begin{columns}
            \begin{column}{0.7\textwidth}
                \begin{itemize}
                    \item \textbf{Nicolin Dora}
                    \begin{itemize}
                        \item Product Owner \& Developer
                    \end{itemize}
                    \item \textbf{Abidin Vejseli}
                    \begin{itemize}
                        \item Scrummaster \& Developer
                    \end{itemize}
                    \item \textbf{Kilian Wampfler} Developer
                    \begin{itemize}
                        \item Developer
                    \end{itemize}
                \end{itemize}
            \end{column}
            \begin{column}{0.3\textwidth}
                \includegraphics[height=2cm]{pictures/dora_2} \\
                %\caption{Nicolin Dora}
                \vspace{1em}
                \includegraphics[height=2cm]{pictures/vejseli_2} \\
                %\caption{Abidin Vejseli}
                \vspace{1em}
                \includegraphics[height=2cm]{pictures/wampfler_2} \\
                %\caption{Kilian Wampfler}
                \vspace{1em}
            \end{column}
        \end{columns}
    \end{frame}

    \begin{frame}{Other Roles}
        \begin{columns}
            \begin{column}{0.7\textwidth}
                \begin{itemize}
                    \item \textbf{Dr. Simon Kramer}
                    \begin{itemize}
                        \item Stakeholder \& Client
                    \end{itemize}
                    \item \textbf{Frank Helbling}
                    \begin{itemize}
                        \item PM-Advisor
                    \end{itemize}
                \end{itemize}
            \end{column}
            \begin{column}{0.3\textwidth}
                \includegraphics[height=2cm]{pictures/kramer_2} \\
                %\caption{Nicolin Dora}
                \vspace{1em}
                \includegraphics[height=2cm]{pictures/helbling_2} \\
                %\caption{Abidin Vejseli}
                \vspace{1em}
            \end{column}
        \end{columns}
    \end{frame}

    \begin{frame}{Our Scrum Adaptions}
        \framesubtitle{Sprint}
        \includegraphics[width=1\textwidth]{pictures/scrum_adaptions-Sprint}
    \end{frame}

    \begin{frame}{Our Scrum Adaptions}
        \framesubtitle{Sprint Planning}
        \includegraphics[width=1\textwidth]{pictures/scrum_adaptions-Sprint_Planning}
    \end{frame}

    \begin{frame}{Our Scrum Adaptions}
        \framesubtitle{Daily Scrum}
        \includegraphics[width=1\textwidth]{pictures/scrum_adaptions-Daily_Scrum}
    \end{frame}

    \begin{frame}{Our Scrum Adaptions}
        \framesubtitle{Sprint Review}
        \includegraphics[width=1\textwidth]{pictures/scrum_adaptions-Sprint_Review}
    \end{frame}

    \begin{frame}{Our Scrum Adaptions}
        \framesubtitle{Sprint Retro}
        \includegraphics[width=1\textwidth]{pictures/scrum_adaptions-Sprint_Retro}
    \end{frame}

    \begin{frame}{Estimation Method}
        \framesubtitle{T-Shirt Sizes}
        \begin{center}
            \includegraphics[width=0.6\textwidth]{pictures/tshirt_sizes}
        \end{center}
    \end{frame}

    \begin{frame}{Velocity}
        \begin{center}
            \includegraphics[width=0.8\textwidth]{pictures/velocity}
        \end{center}
    \end{frame}

    \begin{frame}{Backlog}
        \begin{center}
            \includegraphics[width=0.9\textwidth]{pictures/backlog}
        \end{center}
    \end{frame}

    \begin{frame}{Epics}
        \begin{center}
            \includegraphics[width=0.8\textwidth]{pictures/epics}
        \end{center}
    \end{frame}

    \begin{frame}{Spint 2}
        \framesubtitle{Sprint Goal}
        \includegraphics[width=1\textwidth]{pictures/sprint_2_goal}
    \end{frame}

    \begin{frame}{Spint 2}
        \framesubtitle{Categories}
        \includegraphics[width=1\textwidth]{pictures/sprint_2_cat}
    \end{frame}

    \begin{frame}{User Story}
        \framesubtitle{Description, Acceptance Criteria, DOR and DOD}
        \begin{columns}
            \begin{column}{0.6\textwidth}
                \begin{center}
                    \includegraphics[width=0.8\textwidth]{pictures/user_story_1}
                \end{center}
            \end{column}
            \begin{column}{0.4\textwidth}
                \begin{center}
                    \includegraphics[width=0.7\textwidth]{pictures/user_story_2}
                \end{center}
            \end{column}
        \end{columns}
    \end{frame}

    \begin{frame}{Our learnings}
        \framesubtitle{Categories}
        \begin{center}
            \includegraphics[width=0.5\textwidth]{pictures/scrum_adaptions-learnings}
        \end{center}
    \end{frame}


    %----------------------------------------%


    \section{Questions?}
    \setbeamertemplate{section page}[BFH-ruled]
    \frame{\sectionpage}

    %-------------------------------%


%These can be automaticlly called by using \AtBeginSection{\sectionpage}

%Change base color scheme (option can be added)

    %\setbeamercolor{BFH}{parent=BFH-Orange}

    %\frame{\sectionpage}

    %\setbeamertemplate{section page}[BFH]

    %\frame{\sectionpage}

\end{document}

