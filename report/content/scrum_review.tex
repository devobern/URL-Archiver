\section{Review}

\subsection{Product goal}


\subsection{Sprint goals}

\subsection{Delimitation}


\subsection{Delivery objects}


\subsection{Sprint Backlogs}


\subsection{Product Backlog}


\section{Retrospective I: Scrum Method}

This chapter reflects on the distribution and implementation of Scrum roles as previously described.
It also considers our insights from Scrum Events and Scrum Artifacts.

\section{Scrum Roles}

Nicolin fulfilled the role of Product Owner exceptionally well.
He consistently maintained and prioritized the Product Backlog.
Moreover, he ensured we were in close communication with our customer, Mr. Kramer, guaranteeing that stakeholder requirements were discussed and incorporated.
This was instrumental in the business success of the product being developed.

Abidin, as Scrum Master, executed his responsibilities with diligence, evident in the efficient Scrum operation of the team.
Acting as a liaison between the Product Owner and the Developers, any questions regarding the agile methodology were promptly clarified.
Additionally, obstacles beyond the team's capacity were resolved with the involvement of our coaches.

Kilian held the role of developer and played a key role in shaping the product.
As there were only three of us in the project team, Abidin and Nicolin also contributed to the technical development.
Our goal to satisfy our client, Mr. Simon Kramer, was achieved through the early and continuous delivery of our software.
This ensured the software developed matched Mr. Kramer's vision and requirements.
Moreover, we were able to implement new requirements during the project, as clearly demonstrated by integrating the Wayback Machine.
Throughout the project, we maintained self-organization, a consistent pace, reflection, and expeditious work.


\section{Scrum Events and Scrum Artifacts}

\section{Scrum Events}

\subsection{Sprint}
Our two-week sprint cycles have been effective in balancing workload and facilitating regular feedback.
Setting SMART goals for each sprint provided us with a clear and focused direction and contributed significantly to our team's progress.
Although we occasionally faced challenges in estimating workload, these instances offered valuable lessons in capacity planning.
Exploring more precise estimation techniques could improve our ability to align tasks with our team's capacity, building on our already solid sprint planning process.

\subsection{Sprint Planning}
Sprint planning was a strong point for our team, particularly with the upfront estimation of user stories and prioritization based on business value and team velocity.
We ensured that the most impactful tasks were addressed each sprint.
Encountering larger-than-expected user stories at times provided learning opportunities for better task evaluation.
A more comprehensive review process for estimating user stories could bolster our sprint planning further.

\subsection{Daily Scrum}
Adjusting to bi-weekly meetings complemented our part-time work schedules and kept team communication robust.
Using Microsoft Teams for meetings brought added flexibility and convenience.
While the format sometimes delayed the addressing of immediate issues, it generally promoted focused and efficient discussions.
A mid-week check-in could enhance our responsiveness without burdening the team's schedule.

\subsection{Sprint Review}
Our sprint reviews were invaluable, offering objective evaluations that kept product increments in line with customer needs and expectations.
Responding to customer feedback, though sometimes challenging, was a dynamic learning opportunity that led to significant product enhancements.
More frequent reviews during the sprint could make the feedback integration process even smoother.

\subsection{Sprint Retrospective}
The sprint retrospectives were characterized by constructive, open discussions that effectively pinpointed our successes and growth areas.
These sessions were instrumental in fostering a culture of continuous improvement, with actionable steps consistently identified to enhance team processes.
While implementing these improvements took effort, the retrospectives were central to our team's development and unity.
A structured follow-up mechanism for retrospective suggestions could maximize these sessions' impact.

Although we attempted to provide each other with honest and constructive feedback, we realised that it can be challenging to do so.
This is often due to fear of the recipient's reaction.
However, we believe that with time, giving feedback will become easier for a team that has worked together for an extended period.


\section{Scrum Artifacts}

\subsection{Product Backlog}
The Product Backlog was a dynamic, well-organized tool that effectively captured our project's requirements and priorities.
Regular updates and refinements ensured it remained in sync with our evolving project goals and customer needs.
Clear categorization and prioritization of items significantly aided our planning and decision-making.
We are considering incorporating more frequent stakeholder feedback sessions to ensure that the backlog continues to reflect the most current and relevant project needs.

\subsection{Sprint Backlog}
Our Sprint Backlog was used effectively to map out user stories, providing a clear outline of the sprint goals.
Generally, we managed without creating detailed tasks for each story, favoring a broader view for simplicity and clarity.
Introducing specific tasks for user stories in future sprints could improve granularity and task management.
This approach could enhance tracking and aid in early obstacle identification.

\subsection{Product Increment}
Our approach to product increments has highlighted our team's dedication to delivering high-quality results at the end of each sprint.
The increments met the set criteria and client expectations, showcasing our ability to effectively transform backlog items into tangible, valuable product features.
Regular reviews and refinements ensured alignment with project goals and customer needs.


\section{Retrospektive II: Tools/Instrumente}

\section{Erkenntnisse Tool-Einsatz}
During the project, we used various tools that were essential to our success.
In this section, we will discuss these tools and our experiences with them.

\subsubsection{Java}
Java, our chosen primary programming language, facilitated a smooth development process due to the team's pre-existing knowledge.
This familiarity allowed us to focus on delivering a high-quality application efficiently.

\subsubsection{Maven}
Maven's role in our project was critical, handling compilation, dependency management, and the building process.
Our prior experience with Maven enabled its seamless integration into our workflow, contributing to a smooth development cycle.

\subsubsection{JUnit 5}
JUnit 5 played a pivotal role in our testing framework, guaranteeing the stability and reliability of our application, especially following code refactoring activities, thus ensuring the integrity of our application's logic.

\subsubsection{JetBrains IntelliJ IDEA}
The selection of IntelliJ IDEA as our integrated development environment provided an array of tools that streamlined our development processes, enhancing productivity, and facilitating a high level of code quality.

\subsubsection{GitHub}
GitHub was the backbone of our version control system, offering a robust platform for collaborative code management.
It enabled efficient teamwork and was essential in maintaining a consistent codebase.

\subsubsection{Atlassian Jira}
Atlassian Jira was a cornerstone of our project management, enabling us to track progress, manage priorities, and streamline our workflow effectively, proving itself as an invaluable asset to our project organization.

\subsubsection{LaTeX}
LaTeX was used for creating our project documentation and presentations.
Despite initial challenges due to varying levels of experience within the team, it ultimately enabled us to produce professional and consistent documentation.


\section{Communication}

\subsubsection{Microsoft Teams}
With Microsoft Teams, we were able to streamline our communication, enabling an effective platform for meetings and Scrum events that fostered rapid problem resolution and team agility.
The platform's flexibility and immediacy were crucial to our effective problem-solving process.

\subsubsection{Email and BigBlueButton}
For external communications, particularly with coaches, we used email for asynchronous exchanges and BigBlueButton for real-time video conferencing, ensuring clear and direct dialogue with our stakeholders.

\section{Controlling}

\subsubsection{Atlassian Jira}
Jira served as the central control tool for our project, providing a comprehensive overview of project status and task distribution.
It facilitated clear categorisation and prioritisation of tasks, thereby enhancing our workflow and project management.

\subsubsection{Burndown Charts}
We utilized Burndown Charts to monitor our sprint efficiency and to inform future sprint planning.
This allowed for a balanced workload management and realistic goal-setting based on our team's velocity and past performance.


\section{Lessons learned}


\section{Team}


\section{Nicolin Dora}


\section{Abidin Vejseli}


\section{Killian Wampfler}
