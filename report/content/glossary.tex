\newglossaryentry{URL-Archiver}{
	name={URL-Archiver},
	description={\textbf{A software tool designed to extract, archive, and manage URLs found within digital documents.} It supports various file formats and integrates with services like the Wayback Machine for archiving web pages.},
	plural={URL-Archivers}
}


\newglossaryentry{BibTeX}{
  name={BibTeX},
  description={\textbf{Program for the creation of bibliographical references} and directories in \TeX or \LaTeX\, documents},
  plural=BibTeXs
}

\newglossaryentry{URL}{
  name=URL,
  description={A \textbf{URL (Uniform Resource Locator)} is an internet address that directs to a specific resource, like a webpage.},
  plural=URLs
}

\newglossaryentry{SCRUM}{
	name=SCRUM,
	description={A \textbf{framework for agile software development} focusing on iterative progress through sprints and collaborative team efforts.},
	plural=SCRUM
}

\newglossaryentry{Epic}{
	name=Epic,
	description={\textbf{A large body of work in agile development}, broken down into smaller tasks or user stories.},
	plural=Epics
}

\newglossaryentry{UserStory}{
	name={User Story},
	description={\textbf{A short, simple description of a feature from the perspective of the end user}, used in agile development.},
	plural={User Stories}
}

\newglossaryentry{Sprint}{
	name=Sprint,
	description={\textbf{A set period in the SCRUM framework where specific work has to be completed} and made ready for review.},
	plural=Sprints
}

\newglossaryentry{Latex}{
	name=\LaTeX,
	description={\textbf{A high-quality typesetting system;} it includes features designed for the production of technical and scientific documentation.},
	plural=\LaTeX
}

\newglossaryentry{WaybackMachine}{
	name={Wayback Machine},
	description={\textbf{A digital archive of the World Wide Web}, allowing users to see older versions of web pages.},
	plural={Wayback Machine}q
}

\newglossaryentry{ArchiveToday}{
	name={Archive Today},
	description={\textbf{A web archiving service that stores snapshots of web pages} for preservation and retrieval.},
	plural={Archive Today}
}

\newglossaryentry{MVCPattern}{
	name={Model-View-Controller (MVC) pattern},
	description={Model-View-Controller, a \textbf{software design pattern} for implementing user interfaces, data, and controlling logic.},
	plural={MVC Pattern}
}

\newglossaryentry{FactoryPattern}{
	name={Factory Pattern},
	description={A \textbf{design pattern in software development} used to create objects, allowing interfaces to define object creation but letting subclasses alter the type of objects that will be created.},
	plural={Factory Pattern}
}

\newglossaryentry{FLOSS}{
	name={FLOSS},
	description={\textbf{Free/Libre and Open Source Software (FLOSS)} refers to software that is both free in the sense of freedom (libre) and open source. It allows users the freedom to run, study, modify, and distribute the software for any purpose.},
	plural={FLOSS}
}

\newglossaryentry{CLI}{
	name={CLI},
	description={A \textbf{Command Line Interface (CLI)} is a type of user interface navigated entirely using text commands. It allows users to interact with software or operating systems by typing commands into a console or terminal.},
	plural={CLIs}
}

\newglossaryentry{UnicodeTextFile}{
	name={Unicode Text File},
	description={A \textbf{Unicode Text File is a file that uses the Unicode standard} for encoding its content. Unicode enables the representation of a vast array of characters from various scripts and symbol sets, making it suitable for internationalization and localization.},
	plural={Unicode Text Files}
}

\newglossaryentry{csv}{
	name=CSV,
	description={\textbf{CSV (Comma-Separated Values)} is a file format used to store tabular data, such as a spreadsheet or database. Data fields in CSV files are separated using commas.},
	plural=CSVs
}

\newglossaryentry{GUI}{
	name=GUI,
	description={GUI, or \textbf{Graphical User Interface}, refers to a type of user interface that allows users to interact with electronic devices through graphical icons and visual indicators, as opposed to text-based interfaces, typed command labels, or text navigation. It simplifies the user experience by employing graphical representations of the commands and actions available.},
	plural=GUIs
}


