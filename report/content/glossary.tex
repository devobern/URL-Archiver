\newglossaryentry{BibTeX}{
  name={BibTeX},
  description={Program for the creation of bibliographical references and directories in \TeX or \LaTeX\, documents},
  plural=BibTeXs
}

\newglossaryentry{URL}{
  name=URL,
  description={A URL (Uniform Resource Locator) is an internet address that directs to a specific resource, like a webpage.},
  plural=URLs
}

\newglossaryentry{SCRUM}{
	name=SCRUM,
	description={A framework for agile software development focusing on iterative progress through sprints and collaborative team efforts.},
	plural=SCRUM
}

\newglossaryentry{Epic}{
	name=Epic,
	description={A large body of work in agile development, broken down into smaller tasks or user stories.},
	plural=Epics
}

\newglossaryentry{UserStory}{
	name={User Story},
	description={A short, simple description of a feature from the perspective of the end user, used in agile development.},
	plural={User Stories}
}

\newglossaryentry{Sprint}{
	name=Sprint,
	description={A set period in the SCRUM framework where specific work has to be completed and made ready for review.},
	plural=Sprints
}

\newglossaryentry{Latex}{
	name=\LaTeX,
	description={A high-quality typesetting system; it includes features designed for the production of technical and scientific documentation.},
	plural=\LaTeX
}

\newglossaryentry{WaybackMachine}{
	name={Wayback Machine},
	description={A digital archive of the World Wide Web, allowing users to see older versions of web pages.},
	plural={Wayback Machine}q
}

\newglossaryentry{ArchiveToday}{
	name={Archive Today},
	description={A web archiving service that stores snapshots of web pages for preservation and retrieval.},
	plural={Archive Today}
}

\newglossaryentry{MVCPattern}{
	name={MVC Pattern},
	description={Model-View-Controller, a software design pattern for implementing user interfaces, data, and controlling logic.},
	plural={MVC Pattern}
}

\newglossaryentry{FactoryPattern}{
	name={Factory Pattern},
	description={A design pattern in software development used to create objects, allowing interfaces to define object creation but letting subclasses alter the type of objects that will be created.},
	plural={Factory Pattern}
}
