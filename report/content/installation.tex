\LoadBFHModule{boxes}

\section{Installation (Sysadmin) Manual \& Script} \label{sec::installation_manual}
The URL-Archiver enables the extraction of URLs from any Unicode text or PDF file and allows for interactive archiving
on one of the supported archiving services \index{Archiving Services}.
\begin{bfhWarnBox}
The application was designed to be platform-independent \index{platform-independent}. However, it has only been tested on the following systems, so it cannot be guaranteed to work without restrictions on other platforms.
\begin{itemize}
	\item Windows 11 (Version 23H2)
	\item Windows 10 (Version 22H2)
	\item macOS (Ventura)
	\item Ubuntu (20.04.3 LTS)
\end{itemize}
\end{bfhWarnBox}

\subsection{Requirements}

To build and start the application, ensure that the following dependencies are installed on your system:
\begin{itemize}
	\item Git: Latest stable version recommended.
	\item Maven: Version 3.8 or higher.
	\item Java: Version 21.
\end{itemize}

\subsection{Clone the repository}

To clone the repository, run the following command in a terminal:

\begin{lstlisting}[numbers=none, caption={Command to Clone the Repository for URL-Archiver}, label={lst:git_clone}]
git clone https://github.com/devobern/URL-Archiver.git
\end{lstlisting}


\subsection{Build and run scripts}

The build and run scripts are provided for Windows (\texttt{build.ps1}, \texttt{run.ps1}, \texttt{build\_and\_run.ps1}), Linux, and
MacOS (\texttt{build.sh}, \texttt{run.sh}, \texttt{build\_and\_run.sh}). The scripts are located in the root directory of the project.

\begin{bfhWarnBox}
	The scripts need to be executable. To make them executable, run the following command in a terminal:
	\begin{itemize}
		\item Linux / MacOS: \texttt{chmod +x build.sh run.sh build\_and\_run.sh}
		\item Windows: 
		\begin{itemize}
			\item Open PowerShell as an Administrator. 
			\item Check the current execution policy by running: Get-ExecutionPolicy. 
			\item If the policy is Restricted, change it to RemoteSigned to allow local scripts to run. Execute: Set-ExecutionPolicy RemoteSigned. 
			\item Confirm the change when prompted.
			\item This change allows you to run PowerShell scripts that are written on your local machine. \textbf{Be sure to only run scripts from trusted sources.}
		\end{itemize}
	\end{itemize}
\end{bfhWarnBox}

\subsubsection{Windows}

\paragraph{Build the application}
\mbox{}\\
To build the application, open a command prompt and run the following script:

\begin{lstlisting}[numbers=none, caption={Script to Build the URL-Archiver Application on Windows}, label={lst:build_win}]
./build.ps1
\end{lstlisting}


\paragraph{Run the application}
\mbox{}\\
To run the application, open a command prompt and run the following script:

\begin{lstlisting}[numbers=none, caption={Script to Run the URL-Archiver Application on Windows}, label={lst:run_win}]
./run.ps1
\end{lstlisting}


\paragraph{Build and run the application}
\mbox{}\\
To build and run the application, open a command prompt and run the following script:

\begin{lstlisting}[numbers=none, caption={Script to Build and Run the URL-Archiver Application on Windows}, label={lst:build_run_win}]
./build_and_run.ps1
\end{lstlisting}


\subsubsection{Linux and macOS}

\paragraph{Build the application}
\mbox{}\\
To build the application, open a command prompt and run the following script:

\begin{lstlisting}[numbers=none, caption={Script to Build the URL-Archiver Application on Linux and macOS}, label={lst:build_unix}]
./build.sh
\end{lstlisting}


\paragraph{Run the application}
\mbox{}\\
To run the application, open a command prompt and run the following script:

\begin{lstlisting}[numbers=none, caption={Script to Run the URL-Archiver Application on Linux and macOS}, label={lst:run_unix}]
./run.sh
\end{lstlisting}


\paragraph{Build and run the application}
\mbox{}\\
To build and run the application, open a command prompt and run the following script:

\begin{lstlisting}[numbers=none, caption={Script to Build and Run the URL-Archiver Application on Linux and macOS}, label={lst:build_run_unix}]
./build_and_run.sh
\end{lstlisting}



\section{User Manual}
\begin{bfhWarnBox}
	To follow the instructions in this section, the application must be built. See \ref{sec::installation_manual}.
\end{bfhWarnBox}

The URL-Archiver is a user-friendly application designed for extracting and archiving URLs from text and PDF files. Its intuitive interface requires minimal user input and ensures efficient management of URLs.

\subsection{Getting Started}

\subsubsection{Windows}

Open Command Prompt, navigate to the application's directory, and execute:

\begin{lstlisting}[numbers=none, caption={Script to Run the URL-Archiver Application on Windows (User Manual)}, label={lst:user_run_win}]
./run.ps1
\end{lstlisting}


\subsubsection{Linux / MacOS}

Open Terminal, navigate to the application's directory, and run:

\begin{lstlisting}[numbers=none, caption={Script to Run the URL-Archiver Application on Linux and macOS (User Manual)}, label={lst:user_run_unix}]
./run.sh
\end{lstlisting}




\subsection{Operating Instructions}

Upon launch, provide a path to a text or PDF file, or a directory containing such files. The application will process and display URLs sequentially.

\subsubsection{Navigation}

Use the following keys to navigate through the application:

\begin{itemize}
	\item \textbf{o}: Open the current URL in the default web browser.
	\item \textbf{a}: Access the Archive Menu to archive the URL.
	\item \textbf{s}: Show a list of previously archived URLs.
	\item \textbf{u}: Update and view pending archive jobs.
	\item \textbf{n}: Navigate to the next URL.
	\item \textbf{q}: Quit the application.
	\item \textbf{c}: Change application settings.
	\item \textbf{h}: Access the Help Menu for assistance.
\end{itemize}

\subsubsection{Archiving URLs}

Choose between archiving to Wayback Machine, Archive.today, both services, or canceling.

When opting to use Archive.today for archiving, an automated browser session will initiate, requiring you to complete a captcha. Once resolved, the URL is archived, and the corresponding archived version is then collected and stored within the application.

\subsubsection{Configuration}

Customize Access/Secret Keys and the default browser. Current settings are shown with default values in brackets.

\subsubsection{Exiting}

To exit, press \textbf{q}. If a BibTex file was provided, you'll be prompted to save the archived URLs in the BibTex file. Otherwise, or after saving the URLs in the BibTex file, you'll be prompted to save the archived URLs in a CSV file.

For BibTex entries:
\begin{itemize}
	\item Without an existing note field, URLs are added as: \texttt{note = \{Archived Versions: \textbackslash url\{url1\}, \textbackslash url\{url2\}\}}
	\item With a note field, they're appended as: \texttt{note = \{<current note>, Archived Versions: \textbackslash url\{url1\}, \textbackslash url\{url2\}\}}
\end{itemize}
