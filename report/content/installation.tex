\LoadBFHModule{boxes}

\section{Installation (Sysadmin) Manual \& Script} \label{sec::installation_manual}
The URL-Archiver enables the extraction of URLs from any Unicode text or PDF file and allows for interactive archiving
on one of the supported archiving services \index{Archiving Services}.
\begin{bfhWarnBox}
The application was designed to be platform-independent \index{platform-independent}. However, it has only been tested on the following systems, so it cannot be guaranteed to work without restrictions on other platforms.
\begin{itemize}
	\item Windows 11 (Version 23H2)
	\item Windows 10 (Version 22H2)
	\item macOS (Ventura)
	\item Ubuntu (20.04.3 LTS)
\end{itemize}
\end{bfhWarnBox}

\subsection{Requirements}

To build and start the application, ensure that the following dependencies are installed on your system:
\begin{itemize}
	\item Git: Latest stable version recommended.
	\item Maven: Version 3.8 or higher.
	\item Java: Version 21.
\end{itemize}

\subsection{Clone the repository}

To clone the repository, run the following command in a terminal:

\begin{lstlisting}[numbers=none, caption={Command to Clone the Repository for URL-Archiver}, label={lst:git_clone}]
git clone https://github.com/devobern/URL-Archiver.git
\end{lstlisting}


\subsection{Build and run scripts}

The build and run scripts are provided for Windows (\texttt{build.ps1}, \texttt{run.ps1}, \texttt{build\_and\_run.ps1}), Linux, and
MacOS (\texttt{build.sh}, \texttt{run.sh}, \texttt{build\_and\_run.sh}). The scripts are located in the root directory of the project.

\begin{bfhWarnBox}
	The scripts need to be executable. To make them executable, run the following command in a terminal:
	\begin{itemize}
		\item Linux / MacOS: \texttt{chmod +x build.sh run.sh build\_and\_run.sh}
		\item Windows: 
		\begin{itemize}
			\item Open PowerShell as an Administrator. 
			\item Check the current execution policy by running: Get-ExecutionPolicy. 
			\item If the policy is Restricted, change it to RemoteSigned to allow local scripts to run. Execute: Set-ExecutionPolicy RemoteSigned. 
			\item Confirm the change when prompted.
			\item This change allows you to run PowerShell scripts that are written on your local machine. \textbf{Be sure to only run scripts from trusted sources.}
		\end{itemize}
	\end{itemize}
\end{bfhWarnBox}

\subsubsection{Windows}

\paragraph{Build the application}
\mbox{}\\
To build the application, open a command prompt and run the following script:

\begin{lstlisting}[numbers=none, caption={Script to Build the URL-Archiver Application on Windows}, label={lst:build_win}]
./build.ps1
\end{lstlisting}


\paragraph{Run the application}
\mbox{}\\
To run the application, open a command prompt and run the following script:

\begin{lstlisting}[numbers=none, caption={Script to Run the URL-Archiver Application on Windows}, label={lst:run_win}]
./run.ps1
\end{lstlisting}


\paragraph{Build and run the application}
\mbox{}\\
To build and run the application, open a command prompt and run the following script:

\begin{lstlisting}[numbers=none, caption={Script to Build and Run the URL-Archiver Application on Windows}, label={lst:build_run_win}]
./build_and_run.ps1
\end{lstlisting}


\subsubsection{Linux and macOS}

\paragraph{Build the application}
\mbox{}\\
To build the application, open a command prompt and run the following script:

\begin{lstlisting}[numbers=none, caption={Script to Build the URL-Archiver Application on Linux and macOS}, label={lst:build_unix}]
./build.sh
\end{lstlisting}


\paragraph{Run the application}
\mbox{}\\
To run the application, open a command prompt and run the following script:

\begin{lstlisting}[numbers=none, caption={Script to Run the URL-Archiver Application on Linux and macOS}, label={lst:run_unix}]
./run.sh
\end{lstlisting}


\paragraph{Build and run the application}
\mbox{}\\
To build and run the application, open a command prompt and run the following script:

\begin{lstlisting}[numbers=none, caption={Script to Build and Run the URL-Archiver Application on Linux and macOS}, label={lst:build_run_unix}]
./build_and_run.sh
\end{lstlisting}