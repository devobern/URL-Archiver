\section{Future Work}
The future vision for the URL-Archiver project includes several enhancements to extend its capabilities and improve the user experience.
These potential additions, formulated within our project's limited timeframe, include:

\begin{itemize}
	\item Enhancing the URL extraction process to accurately identify complex URL structures and allow manual adjustments by users.
    \item Adding more archiving services for a broader reach.
    \item Expanding support for various input file types.
    \item Implementing a user-friendly graphical interface.
    \item Enabling multilingual support for global accessibility.
    \item Automatically archiving all URLs in a file for efficiency.
    \item Providing more detailed setting options for user customization.
    \item Publishing the application in package repositories to simplify installation.
    \item Improving the code layout, like breaking up the controller for better clarity.
\end{itemize}

Although the current URL extraction process is a foundational element of the URL-Archiver's architecture, it can be refined and enhanced. Critical steps towards improving the overall efficacy of the system include addressing the challenges in regex-based URL detection and introducing user-driven URL modification capabilities.
Integrating services such as Memento Time Travel would expand our archiving capabilities and provide a wider historical perspective.
Adapting the application to handle various file formats like .docx could increase its relevance in various other settings.
The introduction of a graphical user interface is another key area, which would dramatically improve user interaction, making the tool more accessible and engaging, particularly for those less versed in CLI environments.
Adding multilingual support is also crucial, as it would break down language barriers, enhancing the tool's usability.
The proposed enhancements, such as automatic URL archiving, detailed settings, application publishing in repositories, and code layout improvements like refactoring the controller, collectively aim to not only improve the tool's functionality and user experience but also to ensure better maintainability and ease of future extensions.
