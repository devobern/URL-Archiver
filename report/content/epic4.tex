\subsubsection{Epic 4: Interaction with archive.ph}
\textbf{Goal}: Automate the process of archiving URLs via archive.ph while ensuring user interaction is seamless and all necessary data is captured for later use.

\begin{enumerate}
    \item \textbf{Automated URL Submission}
    \begin{itemize}
        \item \textbf{Description}: As a user, I want the system to automatically fill in the URL into the archive.ph input field and submit it for archiving.
        \item \textbf{Acceptance Criteria}:
        \begin{itemize}
            \item Upon initiation, system opens the archive.ph website in a browser.
            \item System auto-fills the given URL into the appropriate input field.
            \item System automatically triggers the submission process for archiving.
        \end{itemize}
    \end{itemize}

    \item \textbf{User Interaction for Captchas}
    \begin{itemize}
        \item \textbf{Description}: If required, I want to manually solve captchas to ensure the URL gets archived.
        \item \textbf{Acceptance Criteria}:
        \begin{itemize}
            \item If archive.ph presents a captcha, the system allows the user to solve it manually.
            \item The archiving process proceeds once the captcha is successfully solved.
        \end{itemize}
    \end{itemize}

    \item \textbf{Automatic Retrieval of Archived URL}
    \begin{itemize}
        \item \textbf{Description}: Once a URL is archived, I want the system to automatically retrieve and display the archived URL to me.
        \item \textbf{Acceptance Criteria}:
        \begin{itemize}
            \item System captures the new archived URL from archive.ph after the process completes.
            \item The archived URL is displayed to the user immediately.
            \item The archived URL is stored for later processing and reporting.
        \end{itemize}
    \end{itemize}
\end{enumerate}