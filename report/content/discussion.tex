\section{Discussion}
In the course of our academic endeavor, the URL-Archiver project, we have successfully designed and implemented a Java-based application capable of extracting and archiving URLs from Unicode text and PDF documents.
This development demonstrates the practical application of our academic learning in real-world scenarios.
The application is intended to assist professionals in research and journalism by providing a streamlined approach to archive URLs and organize them.

Our project's outcome highlights the application’s proficiency in fulfilling all requirements and reaching the goal to deliver a FLOSS-licensed, platform-independent Java-program called URL-Archiver.
As young professionals in the IT field, our efforts not only gave us valuable hands-on experience, but also contributed to the larger dialogue regarding digital data management.

However, like any academic project, our application is not without its limitations.
Currently, the tool has a limited compatibility with certain file formats, which may limit its applicability.
Moreover, the necessity for manual captcha solving, while ensuring security, does pose an inconvenience and limits the tool's automation capabilities.

Looking ahead, we have a number of enhancements in mind for the URL-Archiver.
Enhancing the application's compatibility with a broader range of file formats would increase its utility.
Furthermore, integrating support for additional archiving services, particularly those with accessible APIs, stands as a significant upgrade.
This would not only automate interactions with a wider range of archiving solutions but also improve both the utility and user experience of the application.

In conclusion, the URL-Archiver effectively serves its intended purpose. However, there is still room for optimisation.
The findings and challenges identified during this project lay the groundwork for future improvements.
We believe that with continued research and development, the URL-Archiver can become an even more versatile and valuable tool.
