\section{Requirements}

\subsection{Epics and User Stories}
In this section, we outline the main features (Epics) of the project and break them down into detailed user tasks (User Stories). This helps provide a clear understanding of the desired functions and behaviors of our software.
\subsubsection{Epic 1: File Input and Processing}
\textbf{Goal}: Allow the user to input various file types via the command line and prepare these files for further processing.
\begin{enumerate}
    \item \textbf{Prompt for File Path Input}
    \begin{itemize}
        \item \textbf{Description}: As a user, when I start the tool, I want to be prompted to input the path to my file, so the tool knows which file to process.
        \item \textbf{Acceptance Criteria}:
        \begin{itemize}
            \item Upon starting the tool, it prompts the user to enter a file path.
            \item On inputting an invalid path or if there are permissions issues, the tool provides a relevant error message.
        \end{itemize}
    \end{itemize}

    \item \textbf{Automatic File Type Detection}
    \begin{itemize}
        \item \textbf{Description}: As a user, I want the tool to automatically detect the file type (based on file extension) and treat it accordingly so that I don't need to specify the file type separately.
        \item \textbf{Acceptance Criteria}:
        \begin{itemize}
            \item The tool automatically identifies if the file is a .BIB, .TEX, .HTML, or .PDF.
            \item For unrecognized file types, the tool provides an appropriate error message.
        \end{itemize}
    \end{itemize}

    \item \textbf{Processing of Directories}
    \begin{itemize}
        \item \textbf{Description}: As a user, I want to input a whole directory, so the tool processes all supported files contained within.
        \item \textbf{Acceptance Criteria}:
        \begin{itemize}
            \item The tool can accept directory paths after the prompt.
            \item It processes all supported file types within the directory.
            \item The tool gives a message if files within the directory are skipped due to their type.
        \end{itemize}
    \end{itemize}

    \item \textbf{Processing Feedback}
    \begin{itemize}
        \item \textbf{Description}: As a user, I want to receive feedback when the tool starts processing the file and when it finishes, to know the status.
        \item \textbf{Acceptance Criteria}:
        \begin{itemize}
            \item A message is displayed when the processing of a file starts.
            \item Upon completion, a confirmation message is shown, which also includes any potential errors or warnings.
        \end{itemize}
    \end{itemize}
\end{enumerate}
\subsubsection{Epic 2: URL Detection and Extraction}
\textbf{Goal}: Accurately detect and extract URLs from input files for further processing.

\begin{enumerate}
    \item \textbf{Scan Files for URLs}
    \begin{itemize}
        \item \textbf{Description}: As a user, I want the system to scan my input files and identify any embedded URLs so that they can be extracted for archiving.
        \item \textbf{Acceptance Criteria}:
        \begin{itemize}
            \item System can detect URLs in a variety of file formats including .BIB, .TEX, .HTML, and .PDF.
            \item Detected URLs are listed without any duplication.
        \end{itemize}
    \end{itemize}

    \item \textbf{Use Regular Expressions for Extraction}
    \begin{itemize}
        \item \textbf{Description}: As a user, I want the system to use regular expressions or other reliable techniques to extract URLs so that all valid URLs are captured without error.
        \item \textbf{Acceptance Criteria}:
        \begin{itemize}
            \item System uses a robust regular expression pattern that matches most URL formats.
            \item Extracted URLs are validated to ensure they are in the correct format.
        \end{itemize}
    \end{itemize}

    \item \textbf{Store URL Line Number or Context}
    \begin{itemize}
        \item \textbf{Description}: As a user, when a URL is detected and extracted, I want the system to also store its line number or contextual information from the original file, enabling precise placement of its archived counterpart later on.
        \item \textbf{Acceptance Criteria}:
        \begin{itemize}
            \item Upon URL detection, the system captures and stores the line number or relevant context of the URL from the source file.
            \item This information is utilized later if archived URLs need to be placed back into the original files.
        \end{itemize}
    \end{itemize}


    \item \textbf{Compile a List of URLs}
    \begin{itemize}
        \item \textbf{Description}: After extraction, I want all URLs to be compiled into a single list, eliminating any duplicates, so that I have a clean list for archiving.
        \item \textbf{Acceptance Criteria}:
        \begin{itemize}
            \item The list contains all the unique URLs found in the input files.
            \item Invalid or broken URLs are flagged or removed from the list.
        \end{itemize}
    \end{itemize}
\end{enumerate}
\subsubsection{Epic 3: Web Browser Integration}
\textbf{Goal}: Seamlessly open detected URLs, one at a time, in a web browser for user verification, and immediately initiate the archiving process upon user decision.

\begin{enumerate}
    \item \textbf{Sequential URL Preview}
    \begin{itemize}
        \item \textbf{Description}: As a user, I want to preview each detected URL in my default browser sequentially to verify its content.
        \item \textbf{Acceptance Criteria}:
        \begin{itemize}
            \item System opens one URL at a time in the default browser.
            \item Immediately after the URL is displayed, the system presents the user with the option to archive.
        \end{itemize}
    \end{itemize}

    \item \textbf{Immediate Archiving Upon Decision}
    \begin{itemize}
        \item \textbf{Description}: After reviewing a URL in the browser, I want to decide if it should be archived. If I decide to archive, the system should immediately initiate the archiving process.
        \item \textbf{Acceptance Criteria}:
        \begin{itemize}
            \item System provides a prompt to accept or decline the archiving of the displayed URL.
            \item If the user chooses to archive, the system directly begins the archiving process, and the user may need to manually solve captchas.
        \end{itemize}
    \end{itemize}

    \item \textbf{Track Archiving Progress}
    \begin{itemize}
        \item \textbf{Description}: As a user, I want a clear indicator of how many URLs have been displayed, archived, and how many are left to process.
        \item \textbf{Acceptance Criteria}:
        \begin{itemize}
            \item The system displays a counter indicating the number of URLs already shown to the user.
            \item Another counter indicates how many URLs have been chosen for archiving.
            \item Yet another counter shows how many URLs remain to be processed/displayed.
        \end{itemize}
    \end{itemize}

    \item \textbf{Store User Decisions for Reporting}
    \begin{itemize}
        \item \textbf{Description}: As a user, after making a decision about archiving each URL, I want the system to store my choices so that they can be referred to or reported on later.
        \item \textbf{Acceptance Criteria}:
        \begin{itemize}
            \item The system maintains a record of each URL and the user's decision (archived or not archived).
            \item The stored decisions are available for any subsequent reporting needs.
        \end{itemize}
    \end{itemize}

\end{enumerate}
\subsubsection{Epic 4: Interaction with archive.ph}
\textbf{Goal}: Automate the process of archiving URLs via archive.ph while ensuring user interaction is seamless and all necessary data is captured for later use.

\begin{enumerate}
    \item \textbf{Automated URL Submission}
    \begin{itemize}
        \item \textbf{Description}: As a user, I want the system to automatically fill in the URL into the archive.ph input field and submit it for archiving.
        \item \textbf{Acceptance Criteria}:
        \begin{itemize}
            \item Upon initiation, system opens the archive.ph website in a browser.
            \item System auto-fills the given URL into the appropriate input field.
            \item System automatically triggers the submission process for archiving.
        \end{itemize}
    \end{itemize}

    \item \textbf{User Interaction for Captchas}
    \begin{itemize}
        \item \textbf{Description}: If required, I want to manually solve captchas to ensure the URL gets archived.
        \item \textbf{Acceptance Criteria}:
        \begin{itemize}
            \item If archive.ph presents a captcha, the system allows the user to solve it manually.
            \item The archiving process proceeds once the captcha is successfully solved.
        \end{itemize}
    \end{itemize}

    \item \textbf{Automatic Retrieval of Archived URL}
    \begin{itemize}
        \item \textbf{Description}: Once a URL is archived, I want the system to automatically retrieve and display the archived URL to me.
        \item \textbf{Acceptance Criteria}:
        \begin{itemize}
            \item System captures the new archived URL from archive.ph after the process completes.
            \item The archived URL is displayed to the user immediately.
            \item The archived URL is stored for later processing and reporting.
        \end{itemize}
    \end{itemize}
\end{enumerate}
\subsubsection{Epic 5: Output and Reporting}
\textbf{Goal}: Provide the user with an organized CSV file detailing URLs and their archived counterparts. Also, allow for integration of archived URLs back into supported input files.

\begin{enumerate}
    \item \textbf{Generate CSV File}
    \begin{itemize}
        \item \textbf{Description}: As a user, I want the system to produce a CSV file containing all original URLs and their corresponding archived URLs.
        \item \textbf{Acceptance Criteria}:
        \begin{itemize}
            \item A CSV file is generated upon completion of the archiving process.
            \item Each row in the CSV contains the original URL and its archived counterpart.
        \end{itemize}
    \end{itemize}

    \item \textbf{Integrate Archived URLs into Supported Files}
    \begin{itemize}
        \item \textbf{Description}: If desired, I want the system to insert the archived URL back into the original file, following its corresponding original URL.
        \item \textbf{Acceptance Criteria}:
        \begin{itemize}
            \item The system recognizes supported file types for this integration process.
            \item Upon user approval, the archived URL is inserted in the appropriate location (e.g., following its original URL) within the file.
        \end{itemize}
    \end{itemize}
\end{enumerate}
\clearpage
\subsection{Functional requirements}
The URL-Archiver project, as described in the original project description, which can be found in the Appendix (\ref{org_proj_desc}), is designed to be a platform-independent Java program that performs several key functions:

\begin{itemize}
    \item \textbf{Input Processing:} It accepts directories or files (including any Unicode-text- and PDF-Files) as input and scans them for URLs.
    \item \textbf{URL Extraction:} The program extracts all URLs using regular expressions.
    \item \textbf{Optional Browser Interaction:} It has an optional feature to open URLs in a web browser.
    \item \textbf{URL Archiving:} The program posts URLs to an archiving service (Archive Today \& The Wayback Machine) and retrieves the archived URLs.
    \item \textbf{Output Generation:} It outputs a CSV file containing the original and archived URL pairs.
    \item \textbf{Optional File Integration:} The program can optionally insert archived URLs into a .BIB file.
\end{itemize}

Additionally, the project emphasizes minimal, modular, and self-explanatory code. The report for the project is expected to be concise and include the original project description.

\subsection{Boundary}
The following limits apply:

\begin{itemize}
    \item \textbf{Input:} The application processes only Unicode-text-files (e.g.: .BIB, .TEX; .HTML; etc.) and PDF-files
    \item \textbf{Output:} The application exports the extracted and archived URLs in either CSV format or directly in a BIB file.
    \item \textbf{Archiving:} At present, the application should only support Archive Today and The Wayback Machine. Please refer to the documentation of the Wayback Machine\footnote{\href{https://archive.org/details/spn-2-public-api-page-docs/mode/2up}{API Documentation Wayback Machine}} or Archive Today\footnote{\href{https://archive.ph/faq}{FAQ of Archive Today}} for any restrictions on their APIs or services.
\end{itemize}
\clearpage
\subsection{Pre-conditions}
For the application to run smoothly, the following pre-conditions must be met:

\begin{itemize}
    \item \textbf{Java Environment:} The application necessitates the installation of Java Runtime Environment (JRE) version 21 on the system where it is intended to be executed.
    \item \textbf{File Formats:} The software is designed to operate with specific file formats, primarily Unicode text files and PDF files. It is necessary to provide these file types for successful operation.
    \item \textbf{Archiving Service Access:} The application requires internet connectivity and access to web archiving services, such as Archive Today, to enable its URL archiving functionality.
    \item \textbf{Login + S3 Credentials (Wayback Machine only):} To archive with the Wayback Machine, a login and S3 credentials are required.
\end{itemize}
\clearpage